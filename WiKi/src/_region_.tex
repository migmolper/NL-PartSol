\message{ !name(../mpm.tex)}\documentclass[10pt,a4paper]{article}
\usepackage[utf8]{inputenc}
\usepackage[english]{babel}
\usepackage{amsmath}
\usepackage{amsfonts}
\usepackage{amssymb}
\usepackage{makeidx}
\usepackage{graphicx}
\usepackage{fourier}
\usepackage{hyperref}
\usepackage[left=2cm,right=2cm,top=2cm,bottom=2cm]{geometry}
\author{Miguel Molinos Pérez}
\title{MPM}
\begin{document}

\message{ !name(SECTIONS/introduccion.tex) !offset(-15) }
\section{Introduction}
\label{sec:intro}

The past several decades have brought tremendous advances in computing
power and provided fertile ground for the development of the
computational sciences. In computational solid mechanics the Finite
Element Method (FEM) (METER CITA MOLONA) has been very successfully
applied to a wide range of problems with good results. However, body
fixed FEM meshes can be difficult and time consuming to generate for
complex three-dimensional objects. Further, mesh distorsion associated
with large deformations compromises solution accuracy, ultimately
requiring re-meshing. These difficulties have spurred the development
of alternative discretization strategies which avoid mesh distorsion by
dicretizing at points and never
maintaining a body-fixed mesh.\\

Quite a number of ''meshless methods'' have been developed. Some of
the ways in which the methods differ include whether or not a
temporary mesh is used in the solution procedure, whether the
discretization procedure begins with the

\cite{Schreyer1994}


%%% Local Variables:
%%% mode: latex
%%% TeX-master: "../mpm"
%%% End:

\message{ !name(../mpm.tex) !offset(6) }

\end{document}

%%% Local Variables:
%%% mode: latex
%%% TeX-master: t
%%% End:
