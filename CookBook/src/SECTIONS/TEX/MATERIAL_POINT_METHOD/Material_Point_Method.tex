
\section{State of art of the material point method}
\label{sec:state-art-material}

The material point method (MPM) was developed by Sulsky and coworkers,
see \cite{Schreyer1994}, as an extension for solids mechanics of the
fluid-implicit particle-method (FLIP) developed by (Brackbill). Further, the
FLIP originally was proposed as an extension of the particle-in-cell
method developed by (Harlow). More recent developments are centered in
trying to solve the well-know grid-crossing error of the MPM,
described deeply in \ref{sec:grid-crossing-errors}, the first of this early attempts was the
presented by \cite{Bardenhagen2004} called the generalized interpolation
material point method (GIMP), other approach to solve this problem are
the implicit Generalized Interpolation Material Point (GIMP) presented
by \cite{Charlton2017}, the convected Particle Domain Interpolation
(CPDI) and the second-order convected particle domain interpolation
(CPDI 2) by \cite{Sadeghirad2013}, \cite{Sadeghirad2013a}, the dual
Domain Material Point Method (DDMPM) by \cite{Zhang2011}, and the
B-Spline MPM by \cite{Bing2019}.

All of this methods has a starting point the variation form for
conservation momentum \eqref{eq:wf_conserv_momentum3} presented in
\ref{sec:deriv_eq}. In order to arrive to a finite set of equations,
the continuum domain $\Omega$ is discretized as as a finite sum of
material points. Here, the index $p$ is used for the material point
values, $I$ for the nodes of the grid nodes and the index $i,j,k$ are
used for the dimensions of the variables. Each material point has a
position $x_{p,i}$ defined by the centroid of the volume $\Omega_p$.

Each material point is assigned initial values of position, velocity,
mass, volume and stress, denoted by $x_{p,i}$, $v_{p,i}$, $m_{p}$, $V_{p,i}$,
$\sigma_{p,ij}$, and others. Given a material point property, $f_p$, a
representation consistent with the initial discretization procedure is
the sum over the material point as we can see in \eqref{eq:mpm_discretization}
\begin{equation}
  \label{eq:mpm_discretization}
  f(x_i,t) = \sum_p^{N_p} f_p(t) \chi_p(x_i)
\end{equation}
where $\chi$ denotes the particle characteristic function.

As an example, the density field can be discretized as we can see in
\eqref{eq:mpm_density}.
\begin{equation}
  \label{eq:mpm_density}
  \rho(x_i,t) =  \sum_p^{N_p}m_p \chi_p(x_i)
\end{equation}
In order to discretize the stress therm of  \eqref{eq:wf_conserv_momentum3}, a specific
stress is defined by \eqref{eq:specific_stress}
\begin{equation}
  \label{eq:specific_stress}
  \sigma_s = \frac{\sigma}{rho}
\end{equation}
so the stress term becomes
\begin{equation}
  \label{eq:specific_stress_intforces}
  \int_{\Omega}  \nabla \psi^T : \sigma\ d\Omega = \int_{\Omega}  \nabla \psi^T : \rho\sigma_s\ d\Omega
\end{equation}

\subsection{The original material point method}
\label{sec:orig-mater-point}

The particle characteristic function are required to be a partition of
unity in the initial configuration as we can see in
(\ref{eq:part_uniti_chi_i}).

\begin{equation}
  \label{eq:part_uniti_chi_i}
  \sum_{p=1}^{Np}\chi_p^i(x)\ =\ 1
\end{equation}

Where $\chi_p^i$ denotes the particle characteristic functions
restricted to their initial positions and undeformed state. In the
simplest cases, particle characteristic function are taken to be
initially non-overlapping. However, nothing precludes overlapping, or
''fuzzy'' particles, as discussed later. Initial particle volumes
$V_p^i$ are defined by (\ref{eq:vpi_def}).
\begin{equation}
  \label{eq:vpi_def}
  V_p^i = \int_{\Omega^i}\chi^i_p(x)\ d\Omega
\end{equation}
where $\Omega^i$ is the initial volume of the continuum body to be
discredited. In addition to initial particle volumes, the material
point initial masses, $m_p^i$, momenta, $p_p^i$ ans stresses,
$\sigma_p^i$, must be defined. These properties of the continuum
against the particle characteristic functions, as we can see in the
equations (\ref{eq:m_pi}) and (\ref{eq:p_pi})

\begin{align}
  \label{eq:m_pi}
  m_p^i =& \int_{\Omega_i}\rho^i(x)\chi_p^i(x)\ d\Omega \\
  \label{eq:p_pi}
  p_p^i =& \int_{\Omega_i}\rho^i(x)v^i(x)\chi_p^i(x)\ d\Omega
\end{align}

where $\rho^i$

\subsection{Discrete Solution Procedure}
\label{sec:discr_sol_proc}

Given a material point property, $f_p$, a representation consistent
with the initial discretization procedure is the sum over the material
points.
\begin{equation}
  \label{eq:mat_point_discretiz}
  f(x) = \sum_p^{Np}f_p \chi_p(x)  
\end{equation}

The particle characteristic functions are used as a basis for
representing particle data throughout the computational domain and
determine the degree of smoothness of the spatial variation.

Approaching term by term of (\ref{eq:wf_conserv_momentum3}) using
(\ref{eq:mat_point_discretiz}) we get the following :



\begin{itemize}
\item Acceleration term :
  \begin{eqnarray}
    \label{eq:wf_conserv_momentum_acc}
   \int_{\Omega} \psi^T \rho \cdot a \cdot d\Omega =& \int_{\Omega} \psi^T \frac{m}{V}
                                                     \dot{v}\ d\Omega = \int_{\Omega} \psi^T \frac{1}{V}
                                                     \overbrace{m\dot{v}}^{\dot{p}}\ d\Omega = \int_{\Omega} \psi^T \frac{\dot{p}}{V} d\Omega = \nonumber \\
  =&  \int_{\Omega} \psi^T \left[ \sum^{Np}_{p=1} \frac{\dot{p}_p}{V_p} \chi_p(x) \right]\ d\Omega =
     \sum^{Np}_{p=1} \left[ \int_{\Omega_p \bigcap \Omega}
     \psi^T\frac{\dot{p}_p}{V_p}\chi_p(x)\ d\Omega \right] 
  \end{eqnarray}


\item Internal forces :
  \begin{align}
    \label{eq:wf_conserv_momentum_int_forces}
    \int_{\Omega}  \nabla \psi^T \sigma\ d\Omega = \int_{\Omega}  \nabla
    \psi^T \left[\sum^{Np}_{p=1} \sigma_p \chi_p(x) \right]\ d\Omega =
    \sum^{Np}_{p=1} \left[ \int_{\Omega_p \bigcap \Omega}
    \nabla\psi^T\sigma_p\chi_p(x)\ d\Omega \right]
  \end{align}
\item External forces :
  \begin{align}
    \label{eq:wf_conserv_momentum_ext_forces}
    \int_{\Omega} \psi^T \rho b\ d\Omega =& \int_{\Omega} \psi^T \frac{m}{V} b\
                                           d\Omega = \int_{\Omega} \psi^T \left[ \sum^{Np}_{p=1}
                                           \frac{m_p}{V_p}\chi_p(x) \right] b\ d\Omega = \sum_{p=1}^{Np}
                                           \left[ \int_{\Omega_p \bigcap \Omega}
                                           \psi^T \frac{m_p}{V_p} \chi_p(x) b\ d\Omega \right]
  \end{align}  
\end{itemize}

Finally we get the \eqref{eq:sf_conserv_momentum} with the GIPM
discretization as \eqref{eq:wf_conserv_momentum_GIMP}:

\begin{align}
  \label{eq:wf_conserv_momentum_GIMP}
  \sum^{Np}_{p=1} \left[  \int_{\Omega_p \bigcap \Omega}
  \psi^T\frac{\dot{p}_p}{V_p}\chi_p(x)\ d\Omega \right] + \sum^{Np}_{p=1} \left[ \int_{\Omega_p \bigcap \Omega}
  \nabla\psi^T\sigma_p\chi_p(x)\ d\Omega \right] =
  \int_{\Gamma}\psi^T t\ d\Gamma + \sum_{p=1}^{Np}
  \left[ \int_{\Omega_p \bigcap \Omega}
  \psi^T \frac{m_p}{V_p} \chi_p(x) b\ d\Omega \right]
\end{align}
\\
where $\Omega_p$ denotes the current support of particle
characteristic function p, and the current particle volumes are
defined by (\ref{eq:charact_volum})

\begin{equation}
  \label{eq:charact_volum}
  V_p = \int_{\Omega_p \bigcap \Omega} \chi_p(x)\ d\Omega
\end{equation}

Rewriting the balance of momentum, the equation 


The other fundamental aspect of PIC methods is the use of a
computational grid. In MPM the grid serves as a scratch pad for the
solution of conservation of momentum, from which particle states are
updated. To complete the discretization procedure, approximations to
the admissible velocity fields, or test functions, are introduced in
terms of grid vertex quantities and grid shape functions. This step is
analogous to the development of FEM discrete equations. However, use
of both grid and particle basis functions to represent test functions
and trial functions, respectively, is a Petrov–Galerkin method,
[Johnson (1987)], and therefore more akin to some of the meshless
methods (in particular [Demkowicz and Oden (1986); Atluri and Zhu
(2000)]) than the FEM. The continuous representation, $g(x)$, of grid
data, $g_i$ , then
\begin{equation}
  g(x) = \sum_{i = 1}^{Nn}=g_iN_i(x)
\end{equation}

Here $N_i(x)$ is a computational grid shape function, which takes unit
value at node $i$ and zero value all the other nodes. Further, the
shape function are required to be a partition of unity

\begin{equation}
  \label{eq:Partition_Unity}
  \sum_{i=1}^{Nn} N_i(x) = 1  
\end{equation}


\begin{eqnarray}  
  \sum^{Np}_{p=1} \left[ \int_{\Omega_p \bigcap \Omega}
  \psi^T\frac{\dot{p}_p}{V_p}\chi_p(x) d\Omega \right] =& \sum^{Np}_{p=1}\left[ \frac{1}{V_p}\int_{\Omega_p \bigcap \Omega} N_i(x) \chi_p(x) d\Omega \dot{p}_p  \right] = \nonumber \\
  =& \sum_{p=1}^{Np} \overline{S}_{ip}\cdot \dot{p}_p = \dot{p}_i
\end{eqnarray}

\begin{eqnarray}
  -\sum^{Np}_{p=1}\left[ \int_{\Omega_p \bigcap \Omega}\nabla\psi^T \sigma_p \chi_p  d\Omega \right] =& -\sum^{Np}_{p=1}\left[ \int_{\Omega_p \bigcap \Omega}\nabla N_i(x) \sigma_p \chi_p  d\Omega \right] = \nonumber \\
  =& - \sum^{Np}_{p=1}\left[ \frac{V_p}{V_p} \int_{\Omega_p \bigcap \Omega}\nabla N_i(x) \chi_p  d\Omega \sigma_p \right] = \nonumber \\
  =& - \sum^{Np}_{p=1}\left[V_p\overline{\nabla S}_{ip} \sigma_p \right] = f_i^{int}
\end{eqnarray}

\begin{equation}
  \int_{\Gamma \equiv \partial \Omega} \psi^T t d\Gamma = \int_{\Gamma \equiv \partial \Omega} N_i(x) t d\Gamma = f_i^t
\end{equation}

\begin{eqnarray}
  \sum_{p=1}^{Np} \left[ \int_{\Omega_p \bigcap \Omega} \psi^T \frac{m_p}{V_p} \chi_p(x) b d\Omega \right] = \sum_{p=1}^{Np} \left[ \int_{\Omega_p \bigcap \Omega} N_i(x)  \frac{m_p}{V_p} \chi_p(x) b d\Omega  \right] =  \sum_{p=1}^{Np} \overline{S}_{ip}b m_p = f_i^b  
\end{eqnarray}


\begin{eqnarray}
  \sum_{p=1}^{Np} \overline{S}_{ip}\cdot \dot{p}_p  =
  -\sum^{Np}_{p=1}\left[V_p\overline{\nabla S}_{ip} \sigma_p \right]
  +&  \int_{\Gamma \equiv \partial \Omega} N_i(x) t d\Gamma \nonumber
     +& \sum_{p=1}^{Np} \overline{S}_{ip}b m_p \nonumber\\
  \Downarrow&  \\
  \dot{p}_p = f_i^{int} + f_i^t + f_i^b \nonumber
\end{eqnarray}

%%% Local Variables:
%%% mode: latex
%%% TeX-master: "../../../mpm"
%%% End:
