
\section{Governing Equations}
\label{sec:deriv_eq}

In  the following  derivation  of the  discrete  equations, bold  face
quantities  indicate tensor,  $\nabla$ id  the gradient  operator, and
$\cdot$,  and $:$  are first  order (vector)  and second  order tensor
contraction, respectively. The subscript $p$ is used to index material
point  variables   and  $i$   grid  vertex  variables.   The  notation
$\sum_p^{Np}$ and $\sum_i^{Np}$  is used to denote  summation over all
material points, and over all grid vertex, respectively.\\

Of  interest  in  solid  mechanics is  the  deformation  and  material
response to a continuous solid  body under prescribed loads and initial
conditions,    as   governed    by    conservation    of   mass    and
momentum.  Conservation of  mass  is satisfied  implicitly by  leaving
discrete particle  masses unchanged throughout a  computation. Here we
develop  the  discrete  version  of conservation  of  momentum,  which
permits  evolution  of  particle  momenta   in  time.  We  consider  a
deformable body acted upon by forces and subjected to either kinematic
or  traction  boundary conditions  every  where  on its  surface.  The
continuum mechanics theory

\begin{itemize}
\item Balance of momentum
\item Compatibility
\item
\end{itemize}

 
We works over the balance of momentum in the strong formulation, as we
can see in \ref{eq:sf_conserv_momentum}

\begin{equation}
  \label{eq:sf_conserv_momentum}
  \rho a = \nabla \cdot(\sigma) + \rho b  
\end{equation}

Multiplying (\ref{eq:sf_conserv_momentum}) by $\psi$ we get
(\ref{eq:sf_conserv_momentum_2})

\begin{equation}
  \label{eq:sf_conserv_momentum_2}
  \psi^T \cdot \rho a = \psi^T \cdot \nabla\cdot(\sigma) + \psi^T \cdot \rho b
\end{equation}

The function $\psi$ belongs to a space $\mathcal{H}$ (vectorial space
with scalar product defined) wich verifies
$a(\ddot{u},\dot{u},u,\psi)=l(\psi) \quad \forall\ \psi\ \in\
\mathcal{H}$. In other hand, in every point of $\Omega$ where we have
imposed the Dirichlet bounday condition,


\begin{equation}
  \label{eq:hilbert_space}
  \mathcal{H} = \{  \psi(x)/\psi \mid_{\Gamma_D} = 0 \quad , \quad \psi \in H^1(\Omega)  \}
\end{equation}

In other words, $\mathcal{H}$ is the space defined by all the
functions $g$, that $ \psi$ and $\nabla \psi$ are FUNCIONES DE
CUADRADO INTEGRABLE in $\Omega$

\begin{equation}
  \label{eq:def_psi}
  \int_{\Omega}\psi\ d\Omega < \infty \quad , \quad \int_{\Omega}\nabla\psi\ d\Omega < \infty
\end{equation}

Integrating (\ref{eq:sf_conserv_momentum_2}) over $\Omega$ we get
(\ref{eq:wf_conserv_momentum_1})

\begin{equation}
  \label{eq:wf_conserv_momentum_1}
  \int_{\Omega} \psi^T \cdot \rho a\ d\Omega = \int_{\Omega} \psi^T \cdot \nabla\cdot(\sigma)\ d\Omega + \int_{\Omega} \psi^T \cdot \rho b\ d\Omega
\end{equation}

integrating by parts the therm of the internal tension in
(\ref{eq:wf_conserv_momentum_1}), we get (\ref{eq:wf_conserv_momentum_2})

\begin{equation}
  \label{eq:wf_conserv_momentum_2}
  \int_{\Omega} \psi^T\cdot \rho a\ d\Omega = \int_{\Omega}  \nabla\cdot \left( \psi^T \cdot
    \sigma \right)\ d\Omega - \int_{\Omega}  \nabla\psi^T : \sigma\ d\Omega
  + \int_{\Omega} \psi^T\cdot b\ d\Omega
\end{equation}

applying the Gauss theorem over (\ref{eq:wf_conserv_momentum_2}), and following the definition of the stress
vector $t = \sigma \cdot n$ where $n$ is the vector orthogonal to a
surface in the solid, finally we get the variational form for
conservation of momentum may be written as
\ref{eq:wf_conserv_momentum3}

\begin{equation}
  \label{eq:wf_conserv_momentum3}
  \int_{\Omega} \psi^T \cdot \rho a\ d\Omega = \int_{\Gamma}  \psi^T
  \cdot \overbrace{\sigma \cdot n}^{t}\ d\Gamma - \int_{\Omega}  \nabla \psi^T : \sigma\ d\Omega
  + \int_{\Omega} \psi^T b\ d\Omega  
\end{equation}


% \subsection{Initial discretization}
% \label{sec:ini_discr}

% The particle characteristic function are required to be a partition of
% unity in the initial configuration as we can see in
% (\ref{eq:part_uniti_chi_i}).

% \begin{equation}
%   \label{eq:part_uniti_chi_i}
%   \sum_{p=1}^{Np}\chi_p^i(x)\ =\ 1
% \end{equation}

% Where $\chi_p^i$ denotes the particle characteristic functions
% restricted to their initial positions and undeformed state. In the
% simplest cases, particle characteristic function are taken to be
% initially non-overlapping. However, nothing precludes overlapping, or
% ''fuzzy'' particles, as discussed later. Initial particle volumes
% $V_p^i$ are defined by (\ref{eq:vpi_def}).
% \begin{equation}
%   \label{eq:vpi_def}
%   V_p^i = \int_{\Omega^i}\chi^i_p(x)\ d\Omega
% \end{equation}
% where $\Omega^i$ is the initial volume of the continuum body to be
% discredited. In addition to initial particle volumes, the material
% point initial masses, $m_p^i$, momenta, $p_p^i$ ans stresses,
% $\sigma_p^i$, must be defined. These properties of the continuum
% against the particle characteristic functions, as we can see in the
% equations (\ref{eq:m_pi}) and (\ref{eq:p_pi})

% \begin{align}
%   \label{eq:m_pi}
%   m_p^i =& \int_{\Omega_i}\rho^i(x)\chi_p^i(x)\ d\Omega \\
%   \label{eq:p_pi}
%   p_p^i =& \int_{\Omega_i}\rho^i(x)v^i(x)\chi_p^i(x)\ d\Omega
% \end{align}

% where $\rho^i$

% \subsection{Discrete Solution Procedure}
% \label{sec:discr_sol_proc}

% Given a material point property, $f_p$, a representation consistent
% with the initial discretization procedure is the sum over the material
% points.
% \begin{equation}
%   \label{eq:mat_point_discretiz}
%   f(x) = \sum_p^{Np}f_p \chi_p(x)  
% \end{equation}

% The particle characteristic functions are used as a basis for
% representing particle data throughout the computational domain and
% determine the degree of smoothness of the spatial variation.

% Approaching term by term of (\ref{eq:wf_conserv_momentum3}) using
% (\ref{eq:mat_point_discretiz}) we get the following :



% \begin{itemize}
% \item Acceleration term :
%   \begin{eqnarray}
%     \label{eq:wf_conserv_momentum_acc}
%    \int_{\Omega} \psi^T \rho \cdot a \cdot d\Omega =& \int_{\Omega} \psi^T \frac{m}{V}
%                                                      \dot{v}\ d\Omega = \int_{\Omega} \psi^T \frac{1}{V}
%                                                      \overbrace{m\dot{v}}^{\dot{p}}\ d\Omega = \int_{\Omega} \psi^T \frac{\dot{p}}{V} d\Omega = \nonumber \\
%   =&  \int_{\Omega} \psi^T \left[ \sum^{Np}_{p=1} \frac{\dot{p}_p}{V_p} \chi_p(x) \right]\ d\Omega =
%      \sum^{Np}_{p=1} \left[ \int_{\Omega_p \bigcap \Omega}
%      \psi^T\frac{\dot{p}_p}{V_p}\chi_p(x)\ d\Omega \right] 
%   \end{eqnarray}


% \item Internal forces :
%   \begin{align}
%     \label{eq:wf_conserv_momentum_int_forces}
%     \int_{\Omega}  \nabla \psi^T \sigma\ d\Omega = \int_{\Omega}  \nabla
%     \psi^T \left[\sum^{Np}_{p=1} \sigma_p \chi_p(x) \right]\ d\Omega =
%     \sum^{Np}_{p=1} \left[ \int_{\Omega_p \bigcap \Omega}
%     \nabla\psi^T\sigma_p\chi_p(x)\ d\Omega \right]
%   \end{align}
% \item External forces :
%   \begin{align}
%     \label{eq:wf_conserv_momentum_ext_forces}
%     \int_{\Omega} \psi^T \rho b\ d\Omega =& \int_{\Omega} \psi^T \frac{m}{V} b\
%                                            d\Omega = \int_{\Omega} \psi^T \left[ \sum^{Np}_{p=1}
%                                            \frac{m_p}{V_p}\chi_p(x) \right] b\ d\Omega = \sum_{p=1}^{Np}
%                                            \left[ \int_{\Omega_p \bigcap \Omega}
%                                            \psi^T \frac{m_p}{V_p} \chi_p(x) b\ d\Omega \right]
%   \end{align}  
% \end{itemize}

% Finally we get the \eqref{eq:sf_conserv_momentum} with the GIPM
% discretization as \eqref{eq:wf_conserv_momentum_GIMP}:

% \begin{align}
%   \label{eq:wf_conserv_momentum_GIMP}
%   \sum^{Np}_{p=1} \left[  \int_{\Omega_p \bigcap \Omega}
%   \psi^T\frac{\dot{p}_p}{V_p}\chi_p(x)\ d\Omega \right] + \sum^{Np}_{p=1} \left[ \int_{\Omega_p \bigcap \Omega}
%   \nabla\psi^T\sigma_p\chi_p(x)\ d\Omega \right] =
%   \int_{\Gamma}\psi^T t\ d\Gamma + \sum_{p=1}^{Np}
%   \left[ \int_{\Omega_p \bigcap \Omega}
%   \psi^T \frac{m_p}{V_p} \chi_p(x) b\ d\Omega \right]
% \end{align}
% \\
% where $\Omega_p$ denotes the current support of particle
% characteristic function p, and the current particle volumes are
% defined by (\ref{eq:charact_volum})

% \begin{equation}
%   \label{eq:charact_volum}
%   V_p = \int_{\Omega_p \bigcap \Omega} \chi_p(x)\ d\Omega
% \end{equation}

% Rewriting the balance of momentum, the equation 


% The other fundamental aspect of PIC methods is the use of a
% computational grid. In MPM the grid serves as a scratch pad for the
% solution of conservation of momentum, from which particle states are
% updated. To complete the discretization procedure, approximations to
% the admissible velocity fields, or test functions, are introduced in
% terms of grid vertex quantities and grid shape functions. This step is
% analogous to the development of FEM discrete equations. However, use
% of both grid and particle basis functions to represent test functions
% and trial functions, respectively, is a Petrov–Galerkin method,
% [Johnson (1987)], and therefore more akin to some of the meshless
% methods (in particular [Demkowicz and Oden (1986); Atluri and Zhu
% (2000)]) than the FEM. The continuous representation, $g(x)$, of grid
% data, $g_i$ , then
% \begin{equation}
%   g(x) = \sum_{i = 1}^{Nn}=g_iN_i(x)
% \end{equation}

% Here $N_i(x)$ is a computational grid shape function, which takes unit
% value at node $i$ and zero value all the other nodes. Further, the
% shape function are required to be a partition of unity

% \begin{equation}
%   \label{eq:Partition_Unity}
%   \sum_{i=1}^{Nn} N_i(x) = 1  
% \end{equation}


% \begin{eqnarray}  
%   \sum^{Np}_{p=1} \left[ \int_{\Omega_p \bigcap \Omega}
%   \psi^T\frac{\dot{p}_p}{V_p}\chi_p(x) d\Omega \right] =& \sum^{Np}_{p=1}\left[ \frac{1}{V_p}\int_{\Omega_p \bigcap \Omega} N_i(x) \chi_p(x) d\Omega \dot{p}_p  \right] = \nonumber \\
%   =& \sum_{p=1}^{Np} \overline{S}_{ip}\cdot \dot{p}_p = \dot{p}_i
% \end{eqnarray}

% \begin{eqnarray}
%   -\sum^{Np}_{p=1}\left[ \int_{\Omega_p \bigcap \Omega}\nabla\psi^T \sigma_p \chi_p  d\Omega \right] =& -\sum^{Np}_{p=1}\left[ \int_{\Omega_p \bigcap \Omega}\nabla N_i(x) \sigma_p \chi_p  d\Omega \right] = \nonumber \\
%   =& - \sum^{Np}_{p=1}\left[ \frac{V_p}{V_p} \int_{\Omega_p \bigcap \Omega}\nabla N_i(x) \chi_p  d\Omega \sigma_p \right] = \nonumber \\
%   =& - \sum^{Np}_{p=1}\left[V_p\overline{\nabla S}_{ip} \sigma_p \right] = f_i^{int}
% \end{eqnarray}

% \begin{equation}
%   \int_{\Gamma \equiv \partial \Omega} \psi^T t d\Gamma = \int_{\Gamma \equiv \partial \Omega} N_i(x) t d\Gamma = f_i^t
% \end{equation}

% \begin{eqnarray}
%   \sum_{p=1}^{Np} \left[ \int_{\Omega_p \bigcap \Omega} \psi^T \frac{m_p}{V_p} \chi_p(x) b d\Omega \right] = \sum_{p=1}^{Np} \left[ \int_{\Omega_p \bigcap \Omega} N_i(x)  \frac{m_p}{V_p} \chi_p(x) b d\Omega  \right] =  \sum_{p=1}^{Np} \overline{S}_{ip}b m_p = f_i^b  
% \end{eqnarray}


% \begin{eqnarray}
%   \sum_{p=1}^{Np} \overline{S}_{ip}\cdot \dot{p}_p  =
%   -\sum^{Np}_{p=1}\left[V_p\overline{\nabla S}_{ip} \sigma_p \right]
%   +&  \int_{\Gamma \equiv \partial \Omega} N_i(x) t d\Gamma \nonumber
%      +& \sum_{p=1}^{Np} \overline{S}_{ip}b m_p \nonumber\\
%   \Downarrow&  \\
%   \dot{p}_p = f_i^{int} + f_i^t + f_i^b \nonumber
% \end{eqnarray}





\subsection{Observations}
\label{sec:observations}




%%% Local Variables:
%%% mode: latex
%%% TeX-master: "../../../mpm"
%%% End:
